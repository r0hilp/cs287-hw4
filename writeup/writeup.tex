
\documentclass[11pt]{article}

\usepackage{common}
\title{HW4: Word Segmentation}
\author{Jeffrey Ling \\ jling@college.harvard.edu \and Rohil Prasad \\ prasad01@college.harvard.edu }
\begin{document}

\maketitle{}
\section{Introduction}

In this assignment, we examine the problem of word segmentation. Given a string of characters without spaces, we want to determine where we can insert spaces to segment the string into words. 

We implement and discuss several approaches to word segmentation in this paper, all trained on a portion of the Penn Treebank. Our first class of models are hidden Markov models with emission distributions given by an n-gram count model and an adaptation of Bengio's neural network language model (NNLM). Our second class of models are recurrent neural networks (RNNs), namely the Long Short Term Memory (LSTM) network and the Gated Recurrent Unit (GRU) network. Furthermore, we attempt to optimize the evaluation algorithms used to construct segmentations given a model. For the Markovian models, we use a dynamic programming algorithm which greatly improves predictive accuracy. We also make a small adjustment to the RNN evaluation algorithm that the performance of these models as well. 

In Section 2, we give a formal description of our problem and establish our notation. In Section 3, we give detailed descriptions of the algorithms listed above. In Section 4, we present our experimental results. In Section 5, we discuss our findings.

\section{Problem Description}

Assume our training corpus consists of a sequence of characters $c_1, c_2, \dots, c_N$ where $c_i$ is drawn from a vocabulary $\mathcal{V}$ for every $i$. We use $<sp> \in \mathcal{V}$ to denote the space character. 

Our training data represents this corpus as a set of pairs $(c_i, y_i)$ for $i \in \{1, \dots, N\}$. The output variable $y_i$ is set to be equal to $1$ if the next character $c_{i+1}$ is not a space, and $2$ if it is a space. By default, we set $y_N = 1$ since $c_N$ is clearly not followed by a space. 

Given a sequence $\mathbf{c}'$ of characters $c'_1c'_2\dots c'_M$, we can define the function $f$ to be the % not sure how 

Our goal is to train a model that will take an input character $c$ and output the probability that the next character is a space. Then, we can construct a model-specific evaluation algorithm 

\subsection{Evaluation}



\section{Model and Algorithms}

\subsection{Hidden Markov Models}

\subsubsection{Count Model}

\subsubsection{Neural Network Linear Model}

\subsection{Recurrent Neural Networks}

\subsubsection{Long Short Term Memory Network}

\subsubsection{Gated Recurrent Unit Network}

\section{Experiments}

\section{Conclusion}

\bibliographystyle{apalike}
\bibliography{writeup}

\end{document}
